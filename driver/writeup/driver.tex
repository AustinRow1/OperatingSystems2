\documentclass[onecolumn,draftclsnofoot, 10pt, compsoc]{IEEEtran}

\usepackage{graphicx}
\usepackage[section]{placeins}
\usepackage{caption}

\usepackage{amssymb}                                         
\usepackage{amsmath}                                         
\usepackage{amsthm}                                

\usepackage{alltt}                                           
\usepackage{float}
\usepackage{color}
\usepackage{url}

\usepackage{balance}
\usepackage[TABBOTCAP, tight]{subfigure}
\usepackage{enumitem}
\usepackage{pstricks, pst-node}
\usepackage{url}
\usepackage{setspace}
\usepackage{xspace}
\usepackage{array}
\usepackage{hyperref}
\usepackage{etoolbox}
\AtBeginEnvironment{quote}{\singlespacing\vspace{-\topsep}\small}

%input{pygments.tex}

\usepackage{geometry}
\geometry{left=0.75in,right=0.75in,top=0.75in,bottom=0.75in}
\parindent = 0.0 in
\parskip = 0.1 in

\usepackage{fancyvrb}
\usepackage{textcomp}

\def \ParSpace{\vspace{.75em}}
\def \GroupNumber{17\xspace}
\def \Class{Operating Systems II\xspace}
\def \School{Oregon State University\xspace}
\def \Professor{Kevin McGrath\xspace}

\newcommand{\cred}[1]{{\color{red}#1}}
\newcommand{\cblue}[1]{{\color{blue}#1}}

\newcommand{\NameSigPair}[1]{
	\par
	\makebox[2.75in][r]{#1} \hfil 	\makebox[3.25in]{\makebox[2.25in]{\hrulefill} \hfill			
		\makebox[.75in]{\hrulefill}}
	\par\vspace{-12pt} \textit{
		\tiny\noindent
		\makebox[2.75in]{} \hfil		
		\makebox[3.25in]{
			\makebox[2.25in][r]{Signature} \hfill	\makebox[.75in][r]{Date}
		}
	}
}



%%%%%%%%%%%%%%%%%%%%%%%%%%%%%%%%%%%%%%%
\begin{document}
	\begin{titlepage}
		\pagenumbering{gobble}
		\begin{singlespace}
			\includegraphics[height=4cm]{coe.eps}
			\hfill  
			\par\vspace{.2in}
			\centering
			\scshape{
				\vspace{.5in}
				\textbf{\Huge\Class}\par
				\large{
					\today \\Spring Term
				}
				\vfill
				{\large Prepared for}\par
				\huge \School\par
				\vspace{5pt}
				{\Large{\Professor}\par}
				{\large Prepared by }\par
				% Group\GroupNumber\par
				\vspace{5pt}
				{\Large
					{Austin Row}\par
					{Lazar Sharipoff}\par
					{Benjamin Richards}\par
				}
				\vspace{20pt}
			}
			\begin{abstract}
				This document describes the steps used by group \GroupNumber to complete Project 3 in which we built a block driver with added crypto support for the Linux kernel for \Class. 
				It also answers the questions given in the assignment description regarding the assignment itself.
			\end{abstract}     
		\end{singlespace}
	\end{titlepage}
	\newpage
	\pagenumbering{arabic}
	\tableofcontents
	% 7. uncomment this (if applicable). Consider adding a page break.
	%\listoffigures
	%\listoftables
	\clearpage
	
	
	\section{Steps To Setup Environment}
	Note: This assumes that the files \textbf{yocto\_config}, \textbf{look\_scheduler.patch}, and \textbf{sstf-iosched.c} which were included in this submission have been copied to your home directory on os2.engr.oregonstate.edu.
	\begin{enumerate}
		\item
		SSH on to the OS2 server: \textit{ssh os2.engr.oregonstate.edu. }
		\item
		Clone the yocto repo: \textit{git clone git://git.yoctoproject.org/linux-yocto}
		\item 
		Change to the correct yocto version with \textit{cd linux-yocto} followed by \textit{git checkout v3.19.2}
		\item 
		\textit{cp \texttildelow/driver.patch ./}
		\item 
		\textit{git apply look\_scheduler.patch}
	\end{enumerate}
	
	
	\section{Steps To Build and Run Kernel}
	\begin{enumerate}
		\item
		Split the terminal into two windows using tmux.
		We will refer to these windows as T1 and T2. 
		\item
		T1: \textit{source /scratch/files/environment-setup-i586-poky-linux} \\
		It will appear that nothing happened, but it actually ran commands to configure the shell so that it’s ready to build the kernel. 
		\item
		T1: \textit{make -j4 all} \\
		This is the command that actually builds the kernel and generates the kernel image that will used to boot the operating system within the VM. 
		If it asks you if you want to build the os2 driver enter "M".
		After doing this, this command may take several minutes. 
		Wait until it is done before going on to the next step.
		\item
		T1: 
		\textit{qemu-system-i386 -gdb tcp::5517 -S -nographic -kernel ./arch/x86/boot/bzImage -drive file=/scratch/spring2018/17/core-image-lsb-sdk-qemux86.ext4 -enable-kvm -usb -localtime --no-reboot --append "root=/dev/hda rw console=ttyS0 elevator=look"} \\
		The terminal should hang now and do nothing. 
		You have booted the VM using the bzImage file that you generated in the previous step. 
		The VM’s CPU is halted which is why nothing is happening. 
		Proceed to the next step.	
		\item
		T2: \textit{gdb -tui}
		\item
		T2: \textit{target remote :5517} \\
		This attaches the current GDB process to the process which is running on port 5517 which is the process running the halted VM.
		\item
		T2 : \textit{continue} \\
		Now the VM in T1 will no longer be hanging. 
		Switch back to T1 and wait for it to ask you to login.
		\item
		T1: \textit{root} \\
		You are now within the VM logged in as the root user
		\item
		T1: \textit{scp (Your server):(path to linux-yocto)/drivers/block/os2driver.ko /home/ } \\
		This copies the .ko file we made during running make earlier to the linux kernel.
		\item
		T1: \textit{insmod /home/os2driver.ko } \\
		This loads the module.
		\item
		T1: \textit{fdisk /dev/os20} \\
		This partitions the module.
		\item
		T1: \textit{n}
		T1: \textit{p}
		T1: \textit{1}
		T1: \textit{1}
		T1: (Press enter)
		T1: \textit{w}\\
		This series of settings sets up the partition.
		\item
		T1: \textit{mkfs.ext2 /dev/os20p1 } \\
		This creates a file system on the partition.
		\item
		T1: \textit{mount /dev/os20p1 /mnt/ } \\
		This mounts the device to the /mnt/ folder. Now writing or reading anything to the /mnt/ folder should be using our driver with it's encryption. You can check this by using "dmesg" to see the kernel alerts our driver makes.
	\end{enumerate}
	
	\clearpage
	\section{Block driver Design}
		We worked primarily in one .c file which contained all the functions needed to run a basic block driver. This file was kept in the linux-yocto/drivers/block directory. Our driver contains os2\_exit and os2\_init functions that register the device and it's structure to the operating system, as well as set up the crypto api for future read/write use. We also have a getgeo function which allows us to use fdisk on this version of the operating system. And we have transfer and request functions for handling the movement of data into the driver. \\
	
\begin{Verbatim}
	
	/*
	* A sample, extra-simple block driver. Updated for kernel 2.6.31.
	*
	* (C) 2003 Eklektix, Inc.
	* (C) 2010 Pat Patterson <pat at superpat dot com>
	* Redistributable under the terms of the GNU GPL.
	*/
	
	
	/*
	* https://www.kernel.org/doc/html/v4.13/crypto/api-skcipher.html#single-block-cipher-api
	* https://stackoverflow.com/questions/8927827/encryption-and-decryption-using-aes-256
	*/
	
	#include <linux/module.h>
	#include <linux/moduleparam.h>
	#include <linux/init.h>
	
	#include <linux/kernel.h> /* printk() */
	#include <linux/fs.h>     /* everything... */
	#include <linux/errno.h>  /* error codes */
	#include <linux/types.h>  /* size_t */
	#include <linux/vmalloc.h>
	#include <linux/genhd.h>
	#include <linux/blkdev.h>
	#include <linux/hdreg.h>
	#include <linux/crypto.h> /**/
	
	MODULE_LICENSE("Dual BSD/GPL");
	
	static int major_num = 0;
	module_param(major_num, int, 0);
	static int logical_block_size = 512;
	module_param(logical_block_size, int, 0);
	static int nsectors = 1024; /* How big the drive is */
	module_param(nsectors, int, 0);
	static unsigned int crypto_key_length = 16;
	module_param(crypto_key_length, int, 0);
	static char *crypto_key = "abcdefghijklmnop";
	module_param(crypto_key, charp, 0);
	/*
	* We can tweak our hardware sector size, but the kernel talks to us
	* in terms of small sectors, always.
	*/
	#define KERNEL_SECTOR_SIZE 512
	
	struct crypto_cipher *tfm = NULL;
	unsigned int cipher_blocksize;
	/*
	* Our request queue.
	*/
	static struct request_queue *Queue;
	
	/*
	* The internal representation of our device.
	*/
	static struct os2_device {
	unsigned long size;
	spinlock_t lock;
	u8 *data;
	struct gendisk *gd;
	} Device;
	
	static void print_memory(char *buffer, unsigned int length) {
	int i = 0;
	for(i = 0; i < length; i++) {
	printk(KERN_CONT "%02X", (unsigned)buffer[i]);
	}
	}
	
	static void print_cipher_result(char *from, char *to, unsigned int transformation_size, int encrypting) {
	transformation_size = 50 < transformation_size ? 50 : transformation_size;
	printk(KERN_DEBUG "**OS2: \"");
	print_memory(from, transformation_size);
	if (encrypting) {
	printk(KERN_CONT "\" encrypted to \"");
	} else {
	printk(KERN_CONT "\" decrypted to \"");
	}
	print_memory(to, transformation_size);
	printk(KERN_CONT "\" (showing first %d characters)\n", transformation_size);
	}
	
	/*
	* Handle an I/O request.
	*/
	static void os2_transfer(struct os2_device *dev, sector_t sector,
	unsigned long nsect, char *buffer, int write) {
	unsigned long offset = sector * logical_block_size;
	unsigned long nbytes = nsect * logical_block_size;
	unsigned int ciphered_bytes = 0;
	char temp_buffer[cipher_blocksize + 1];
	
	printk(KERN_INFO "**OS2: Inside os2_transfer\n");
	if ((offset + nbytes) > dev->size) {
	printk (KERN_NOTICE "os2: Beyond-end write (%ld %ld)\n", offset, nbytes);
	return;
	}
	
	while(ciphered_bytes + cipher_blocksize <= nbytes) {
	if (write) {
	crypto_cipher_encrypt_one(tfm, dev->data + offset + sizeof(char)*ciphered_bytes, buffer + sizeof(char)*ciphered_bytes);
	} else {
	crypto_cipher_decrypt_one(tfm, buffer + sizeof(char)*ciphered_bytes, dev->data + offset + sizeof(char)*ciphered_bytes);
	}
	ciphered_bytes += cipher_blocksize;
	}
	
	if (ciphered_bytes < nbytes) {
	if (write) {
	printk(KERN_ALERT "**OS2: encrypting remaining %d bytes\n", (int)(nbytes-ciphered_bytes));
	crypto_cipher_encrypt_one(tfm, temp_buffer, buffer + sizeof(char)*ciphered_bytes);
	memcpy(dev->data + offset + sizeof(char)*ciphered_bytes, temp_buffer, nbytes-ciphered_bytes);
	} else {
	printk(KERN_ALERT "**OS2: decrypting remaining %d bytes\n", (int)(nbytes-ciphered_bytes));
	crypto_cipher_decrypt_one(tfm, temp_buffer, dev->data + offset + sizeof(char)*ciphered_bytes);
	memcpy(buffer, temp_buffer, nbytes-ciphered_bytes);
	}
	}
	
	if (write) {
	print_cipher_result(buffer, dev->data + offset, nbytes, 1);
	} else {
	print_cipher_result(dev->data + offset, buffer, nbytes, 0);
	}
	}
	
	static void os2_request(struct request_queue *q) {
	struct request *req;
	
	req = blk_fetch_request(q);
	while (req != NULL) {
	if (req == NULL || (req->cmd_type != REQ_TYPE_FS)) {
	printk (KERN_NOTICE "**OS2: Skip non-CMD request\n");
	__blk_end_request_all(req, -EIO);
	continue;
	}
	os2_transfer(&Device, blk_rq_pos(req), blk_rq_cur_sectors(req),
	bio_data(req->bio), rq_data_dir(req));
	if ( ! __blk_end_request_cur(req, 0) ) {
	req = blk_fetch_request(q);
	}
	}
	}
	
	/*
	* The HDIO_GETGEO ioctl is handled in blkdev_ioctl(), which
	* calls this. We need to implement getgeo, since we can't
	* use tools such as fdisk to partition the drive otherwise.
	*/
	int os2_getgeo(struct block_device * block_device, struct hd_geometry * geo) {
	long size;
	
	/* We have no real geometry, of course, so make something up. */
	size = Device.size * (logical_block_size / KERNEL_SECTOR_SIZE);
	geo->cylinders = (size & ~0x3f) >> 6;
	geo->heads = 4;
	geo->sectors = 16;
	geo->start = 0;
	return 0;
	}
	
	/*
	* The device operations structure.
	*/
	static struct block_device_operations os2_ops = {
	.owner  = THIS_MODULE,
	.getgeo = os2_getgeo
	};
	
	static int __init os2_init(void) {
	/*
	* Set up our internal device.
	*/
	printk(KERN_ALERT "**OS2: In OS2 Device Driver INIT\n");
	Device.size = nsectors * logical_block_size;
	spin_lock_init(&Device.lock);
	Device.data = vmalloc(Device.size);
	if (Device.data == NULL)
	return -ENOMEM;
	
	tfm = crypto_alloc_cipher("aes", 0, 0);
	if (IS_ERR(tfm)) {
	printk(KERN_ALERT "**OS2: Error while initializing crypto cipher\n");
	goto out;
	}
	
	if (crypto_cipher_setkey(tfm, crypto_key, crypto_key_length)) {
	printk(KERN_ALERT "**OS2: Error while setting crypto key\n");
	goto out;
	}
	
	cipher_blocksize = crypto_cipher_blocksize(tfm);
	
	/*
	* Get a request queue.
	*/
	Queue = blk_init_queue(os2_request, &Device.lock);
	if (Queue == NULL)
	goto out;
	blk_queue_logical_block_size(Queue, logical_block_size);
	/*
	* Get registered.
	*/
	major_num = register_blkdev(major_num, "os2");
	if (major_num < 0) {
	printk(KERN_WARNING "os2: unable to get major number\n");
	goto out;
	}
	/*
	* And the gendisk structure.
	*/
	Device.gd = alloc_disk(16);
	if (!Device.gd)
	goto out_unregister;
	Device.gd->major = major_num;
	Device.gd->first_minor = 0;
	Device.gd->fops = &os2_ops;
	Device.gd->private_data = &Device;
	strcpy(Device.gd->disk_name, "os20");
	set_capacity(Device.gd, nsectors);
	Device.gd->queue = Queue;
	add_disk(Device.gd);
	
	return 0;
	
	out_unregister:
	unregister_blkdev(major_num, "os2");
	out:
	vfree(Device.data);
	if (tfm) {
	crypto_free_cipher(tfm);
	}
	return -ENOMEM;
	}
	
	static void __exit os2_exit(void)
	{
	printk(KERN_ALERT "**OS2: In OS2 Device Driver EXIT\n");
	del_gendisk(Device.gd);
	put_disk(Device.gd);
	unregister_blkdev(major_num, "os2");
	blk_cleanup_queue(Queue);
	vfree(Device.data);
	if (tfm) {
	crypto_free_cipher(tfm);
	}
	}
	
	module_init(os2_init);
	module_exit(os2_exit);
	
	\end{Verbatim}
	
	
	\section{Questions About Assignment}
	\begin{enumerate}
		\item What do you think the main point of this assignment is? \\
		The main point of this assignment is to get a glimpse of how the linux operating system treats drivers and to get experience on making one ourselves.
		\item How did you personally approach the problem? \\
		First we looked for sample block driver code templates by using Google. We discovered that several functions in the templates we found initially were depreciated and were no longer working. In addition there were several functions that we didn't need for our assignment. Eventually we found a working driver template that we were able to properly mount. The next step was to implement encryption to all incoming traffic and decryption to all out outgoing traffic using the crypto api.
		\item How did you ensure your solution was correct? \\
		The way we checked to see if our driver was working was by seeing if the driver would properly show up and function in the linux kernel. This required finding the .ko device, successfully partitioning it, adding a file system to it and mounting it. Once that was done the way we checked encryption was through kernel messages that would print out to dmesg whenever an encryption or decryption even happened.
		\item What did you learn? \\
		Throughout the course of this assignment we learned:
		\begin{itemize}
			\item
			The process of how drivers are partitioned and mounted to the system.
			\item
			That drivers have multiple ways you can utilize them, including multiple types of file systems depending on the version.
			\item
			How to use the API for crypto.
			\item
			How to see kernel messages printed from the driver using dmesg.
			\item
			How devices are registered to the os with exit and init. Additionally we learned of the default driver functions the os uses.
		\end{itemize}
		\item How should the TA evaluate your work? \\
		Running dmesg once the device is properly mounted (As described how to do earlier) will show all of the kernel messages printed during the encryption and decryption of data written to and read from the device. To trigger these messages you must be sure to write a file to the /mnt/ folder.
		Additionally whether or not the device properly mounts is another way to evaulate our work. You can see a list of running modules by using the  lsmod command.
	\end{enumerate}
	
	
	\section{Version Control Log}
	\begin{tabular}{|p{5cm}|p{5cm}|p{5cm}|}
		\hline
		\textbf{Commit} & \textbf{Author} & \textbf{Description}\\
		\hline
		\textcolor{blue}{\underline{\href[pdfnewwindow=true]{https://github.com/AustinRow1/OperatingSystems2/commit/7b736f88d63c1a131b084a9af25725df5d6ef8b8}{7b736f8}}} & Benjamin Richards & added driver folder\\\hline
		\textcolor{blue}{\underline{\href[pdfnewwindow=true]{https://github.com/AustinRow1/OperatingSystems2/commit/f57302faed8da1904a9248999f70fb19c9d19ad6}{f57302f}}} & Benjamin Richards & added patch file\\\hline\end{tabular}
	
	
	
	
	\section{Work Log}
	\begin{center}
		\begin{tabular}{ |p{2cm}|p{12cm}| }
			\hline
			5/21/2018 & Began reading about block devices from online sources such as \cite{block-drivers} started implementing it. 
			\\ \hline
			5/25/2018 & We eventually found this source to be outdated and instead went with an updated but more barebones template\cite{driver-template}. We also added crypto by looking at the Linux crypto API\cite{crypto-api} and reading up on how to use that with blocks\cite{block-crypto} as well as AES-256\cite{AES-256}\\
			\hline
		\end{tabular}
	\end{center}
	
	
	\bibdata{}
	\bibliographystyle{IEEEtran}
	\bibliography{references}
	
\end{document}
